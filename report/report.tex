\include{preamble}






\begin{document}
  \maketitle

%\includegraphics[scale=0.55]{frontpage}

  \section{Results}
  
  
  
  

  The following table summarizes our result:
  
    \bigskip\noindent
  \begin{tabular}{lr}
  	\toprule
  	Input file & MST total weight \\ \midrule
  	USA-highway-miles.txt	 & 16598.0 \\
  	tinyEWG-alpha.txt & 181 \\ \bottomrule
  \end{tabular}
  
    \bigskip
  
     The MST we found in tinyEWG-alpha.txt can be drawn like this:
     
     


  
  \section{Implementation details}
  
  We consider a map, where the coordinates come from loading the strings form the in-file into a String array.

  We talked about two ways to implement a solution:
  
  \begin{itemize}
  	\item starting in an S-field and pushing all legal fields from there to a queue and then pop the fields from the queue and push legal fields from these to the queue until a F-field is reached, counting each move and saving the smallest value for print in the end.
  	\item doing a recursive method that starts in a S-field and calls itself on all legal fields.
  	
 \end{itemize}	
 
 Pros and cons on each method made us decide on the first.
 
 The argument is as follows:
 
 If we store all visited fields with the speed we have visited them with in a simple table, we can check if we have been in a specific position with a specific speed (given by two coordinates) - if this is the case we do not go to this field again. This works because the algorithm gets us to a specific field in the shortes posible route.
  
  

\end{document}
